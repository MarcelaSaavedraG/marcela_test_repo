% Options for packages loaded elsewhere
\PassOptionsToPackage{unicode}{hyperref}
\PassOptionsToPackage{hyphens}{url}
%
\documentclass[
]{article}
\author{Marcela Saavedra Gonzalez}
\date{}

\usepackage{amsmath,amssymb}
\usepackage{lmodern}
\usepackage{iftex}
\ifPDFTeX
  \usepackage[T1]{fontenc}
  \usepackage[utf8]{inputenc}
  \usepackage{textcomp} % provide euro and other symbols
\else % if luatex or xetex
  \usepackage{unicode-math}
  \defaultfontfeatures{Scale=MatchLowercase}
  \defaultfontfeatures[\rmfamily]{Ligatures=TeX,Scale=1}
\fi
% Use upquote if available, for straight quotes in verbatim environments
\IfFileExists{upquote.sty}{\usepackage{upquote}}{}
\IfFileExists{microtype.sty}{% use microtype if available
  \usepackage[]{microtype}
  \UseMicrotypeSet[protrusion]{basicmath} % disable protrusion for tt fonts
}{}
\makeatletter
\@ifundefined{KOMAClassName}{% if non-KOMA class
  \IfFileExists{parskip.sty}{%
    \usepackage{parskip}
  }{% else
    \setlength{\parindent}{0pt}
    \setlength{\parskip}{6pt plus 2pt minus 1pt}}
}{% if KOMA class
  \KOMAoptions{parskip=half}}
\makeatother
\usepackage{xcolor}
\IfFileExists{xurl.sty}{\usepackage{xurl}}{} % add URL line breaks if available
\IfFileExists{bookmark.sty}{\usepackage{bookmark}}{\usepackage{hyperref}}
\hypersetup{
  pdfauthor={Marcela Saavedra Gonzalez},
  hidelinks,
  pdfcreator={LaTeX via pandoc}}
\urlstyle{same} % disable monospaced font for URLs
\usepackage[margin=1in]{geometry}
\usepackage{color}
\usepackage{fancyvrb}
\newcommand{\VerbBar}{|}
\newcommand{\VERB}{\Verb[commandchars=\\\{\}]}
\DefineVerbatimEnvironment{Highlighting}{Verbatim}{commandchars=\\\{\}}
% Add ',fontsize=\small' for more characters per line
\usepackage{framed}
\definecolor{shadecolor}{RGB}{248,248,248}
\newenvironment{Shaded}{\begin{snugshade}}{\end{snugshade}}
\newcommand{\AlertTok}[1]{\textcolor[rgb]{0.94,0.16,0.16}{#1}}
\newcommand{\AnnotationTok}[1]{\textcolor[rgb]{0.56,0.35,0.01}{\textbf{\textit{#1}}}}
\newcommand{\AttributeTok}[1]{\textcolor[rgb]{0.77,0.63,0.00}{#1}}
\newcommand{\BaseNTok}[1]{\textcolor[rgb]{0.00,0.00,0.81}{#1}}
\newcommand{\BuiltInTok}[1]{#1}
\newcommand{\CharTok}[1]{\textcolor[rgb]{0.31,0.60,0.02}{#1}}
\newcommand{\CommentTok}[1]{\textcolor[rgb]{0.56,0.35,0.01}{\textit{#1}}}
\newcommand{\CommentVarTok}[1]{\textcolor[rgb]{0.56,0.35,0.01}{\textbf{\textit{#1}}}}
\newcommand{\ConstantTok}[1]{\textcolor[rgb]{0.00,0.00,0.00}{#1}}
\newcommand{\ControlFlowTok}[1]{\textcolor[rgb]{0.13,0.29,0.53}{\textbf{#1}}}
\newcommand{\DataTypeTok}[1]{\textcolor[rgb]{0.13,0.29,0.53}{#1}}
\newcommand{\DecValTok}[1]{\textcolor[rgb]{0.00,0.00,0.81}{#1}}
\newcommand{\DocumentationTok}[1]{\textcolor[rgb]{0.56,0.35,0.01}{\textbf{\textit{#1}}}}
\newcommand{\ErrorTok}[1]{\textcolor[rgb]{0.64,0.00,0.00}{\textbf{#1}}}
\newcommand{\ExtensionTok}[1]{#1}
\newcommand{\FloatTok}[1]{\textcolor[rgb]{0.00,0.00,0.81}{#1}}
\newcommand{\FunctionTok}[1]{\textcolor[rgb]{0.00,0.00,0.00}{#1}}
\newcommand{\ImportTok}[1]{#1}
\newcommand{\InformationTok}[1]{\textcolor[rgb]{0.56,0.35,0.01}{\textbf{\textit{#1}}}}
\newcommand{\KeywordTok}[1]{\textcolor[rgb]{0.13,0.29,0.53}{\textbf{#1}}}
\newcommand{\NormalTok}[1]{#1}
\newcommand{\OperatorTok}[1]{\textcolor[rgb]{0.81,0.36,0.00}{\textbf{#1}}}
\newcommand{\OtherTok}[1]{\textcolor[rgb]{0.56,0.35,0.01}{#1}}
\newcommand{\PreprocessorTok}[1]{\textcolor[rgb]{0.56,0.35,0.01}{\textit{#1}}}
\newcommand{\RegionMarkerTok}[1]{#1}
\newcommand{\SpecialCharTok}[1]{\textcolor[rgb]{0.00,0.00,0.00}{#1}}
\newcommand{\SpecialStringTok}[1]{\textcolor[rgb]{0.31,0.60,0.02}{#1}}
\newcommand{\StringTok}[1]{\textcolor[rgb]{0.31,0.60,0.02}{#1}}
\newcommand{\VariableTok}[1]{\textcolor[rgb]{0.00,0.00,0.00}{#1}}
\newcommand{\VerbatimStringTok}[1]{\textcolor[rgb]{0.31,0.60,0.02}{#1}}
\newcommand{\WarningTok}[1]{\textcolor[rgb]{0.56,0.35,0.01}{\textbf{\textit{#1}}}}
\usepackage{graphicx}
\makeatletter
\def\maxwidth{\ifdim\Gin@nat@width>\linewidth\linewidth\else\Gin@nat@width\fi}
\def\maxheight{\ifdim\Gin@nat@height>\textheight\textheight\else\Gin@nat@height\fi}
\makeatother
% Scale images if necessary, so that they will not overflow the page
% margins by default, and it is still possible to overwrite the defaults
% using explicit options in \includegraphics[width, height, ...]{}
\setkeys{Gin}{width=\maxwidth,height=\maxheight,keepaspectratio}
% Set default figure placement to htbp
\makeatletter
\def\fps@figure{htbp}
\makeatother
\setlength{\emergencystretch}{3em} % prevent overfull lines
\providecommand{\tightlist}{%
  \setlength{\itemsep}{0pt}\setlength{\parskip}{0pt}}
\setcounter{secnumdepth}{-\maxdimen} % remove section numbering
\ifLuaTeX
  \usepackage{selnolig}  % disable illegal ligatures
\fi

\begin{document}

{
\setcounter{tocdepth}{2}
\tableofcontents
}
\begin{Shaded}
\begin{Highlighting}[]
\FunctionTok{library}\NormalTok{(tidyverse)         }\CommentTok{\# for graphing and data cleaning}
\FunctionTok{library}\NormalTok{(lubridate)         }\CommentTok{\# for working with dates}
\CommentTok{\# For the garden data, you need to first install the remotes library, if you haven\textquotesingle{}t already}
\CommentTok{\# Then, install the gardenR library, if you haven\textquotesingle{}t already. Do this by uncommenting the code below (delete the hashtag) and running it. Then, you should delete this line of code or add the hashtag back so you don\textquotesingle{}t reinstall each time.}
\NormalTok{remotes}\SpecialCharTok{::}\FunctionTok{install\_github}\NormalTok{(}\StringTok{"llendway/gardenR"}\NormalTok{)}
\CommentTok{\# Once the library is installed, you don\textquotesingle{}t need to install it again, but each time you need to load the library using the code below. You will know if you haven\textquotesingle{}t installed the library if the code below produces an error.}
\FunctionTok{library}\NormalTok{(gardenR)}
\CommentTok{\# theme\_set(theme\_minimal())  \# set a theme if desired}
\end{Highlighting}
\end{Shaded}

\begin{Shaded}
\begin{Highlighting}[]
\CommentTok{\# load the garden data }
\FunctionTok{data}\NormalTok{(garden\_harvest)}
\end{Highlighting}
\end{Shaded}

Explain the question you hope to answer and create the graph below in
the Graph Week 1 section. For the first week, you may not have the
skills quite yet to create exactly what you want to create - that's ok!
Get as close as you can, and the instructors will give you feedback to
help you out. You can summarize the data in any way you'd like. Add R
code chunks and comment code as needed. As the weeks go by, you will
continue to build on this file by putting new code in the next Graph
sections. You will keep all the old code, add your instructor's feedback
by copying and pasting it from moodle (found in the Grade section of
moodle), and make improvements from my suggestions and from other ideas
you have. Having the old code and graphs and the instructor feedback
will help you (and the instructors) easily see the progress throughout
the course.

FYI, Prof.~Lisa just added 2021 data to the \texttt{gardenR} package. If
you want to use that data (either with the 2020 data or by itself), you
will need to reinstall the package. Just a warning that variable names
were maintained but names of vegetables and varieties changed in a few
cases over the two years (oops). After you load the dataset, you can
search for gardenR in the Help tab to find out more about the new
datasets.

\hypertarget{graph-week-1}{%
\subsection{Graph Week 1}\label{graph-week-1}}

Question I hope to answer: How does the harvest change per week?

I want to see whats the influence of season in the amount harvested in
the Garden! I expect the harvest to be lower in winter, however, I do
not know enough about harvesting to be sure about that answer. This
graph will help me learn more.

\begin{Shaded}
\begin{Highlighting}[]
\NormalTok{garden\_harvest }\SpecialCharTok{\%\textgreater{}\%} 
  \FunctionTok{mutate}\NormalTok{(}\AttributeTok{Week =} \FunctionTok{week}\NormalTok{(date)) }\SpecialCharTok{\%\textgreater{}\%} 
  \FunctionTok{group\_by}\NormalTok{(Week) }\SpecialCharTok{\%\textgreater{}\%} 
  \FunctionTok{summarise}\NormalTok{(}\AttributeTok{lbs\_per\_Week=}\NormalTok{ (}\FunctionTok{sum}\NormalTok{(weight))}\SpecialCharTok{*}\FloatTok{0.00220462}\NormalTok{)}
\end{Highlighting}
\end{Shaded}

\begin{verbatim}
## # A tibble: 20 x 2
##     Week lbs_per_Week
##    <dbl>        <dbl>
##  1    23        0.179
##  2    24        0.423
##  3    25        3.79 
##  4    26        8.11 
##  5    27        9.88 
##  6    28       12.6  
##  7    29       19.1  
##  8    30       30.3  
##  9    31       44.4  
## 10    32       59.1  
## 11    33       73.8  
## 12    34       85.1  
## 13    35      132.   
## 14    36       58.3  
## 15    37       15.5  
## 16    38      174.   
## 17    39       31.9  
## 18    40       13.3  
## 19    41       89.7  
## 20    42       92.2
\end{verbatim}

\begin{Shaded}
\begin{Highlighting}[]
\NormalTok{harvest\_per\_week }\OtherTok{\textless{}{-}}\NormalTok{ garden\_harvest }\SpecialCharTok{\%\textgreater{}\%} 
  \FunctionTok{mutate}\NormalTok{(}\AttributeTok{Week =} \FunctionTok{week}\NormalTok{(date)) }\SpecialCharTok{\%\textgreater{}\%} 
  \FunctionTok{group\_by}\NormalTok{(Week) }\SpecialCharTok{\%\textgreater{}\%} 
  \FunctionTok{summarise}\NormalTok{(}\AttributeTok{weight\_per\_week\_lbs=}\NormalTok{ (}\FunctionTok{sum}\NormalTok{(weight))}\SpecialCharTok{*}\FloatTok{0.00220462}\NormalTok{) }

\NormalTok{harvest\_per\_week}\SpecialCharTok{\%\textgreater{}\%} 
  \FunctionTok{ggplot}\NormalTok{(}\FunctionTok{aes}\NormalTok{(}\AttributeTok{x=}\NormalTok{Week,}\AttributeTok{y=}\NormalTok{weight\_per\_week\_lbs))}\SpecialCharTok{+}
  \FunctionTok{geom\_col}\NormalTok{(}\AttributeTok{fill=}\StringTok{"lightblue"}\NormalTok{)}\SpecialCharTok{+}
  \FunctionTok{labs}\NormalTok{(}\AttributeTok{y=}\StringTok{"Weight per week (lbs)"}\NormalTok{,}\AttributeTok{x=}\StringTok{"Weeks (starting in June)"}\NormalTok{)}\SpecialCharTok{+}
  \FunctionTok{theme\_linedraw}\NormalTok{()}
\end{Highlighting}
\end{Shaded}

\includegraphics{perfect_garden_graph_files/figure-latex/unnamed-chunk-1-1.pdf}

Instructor's feedback: (copy the feedback from moodle here)

\hypertarget{graph-week-2}{%
\subsection{Graph Week 2}\label{graph-week-2}}

Question I hope to answer: (probably the same as previous week but may
change slightly)

Instructor's feedback: (copy the feedback from moodle here)

\hypertarget{graph-week-3}{%
\subsection{Graph Week 3}\label{graph-week-3}}

Question I hope to answer: (probably the same as previous week but may
change slightly)

Instructor's feedback: (copy the feedback from moodle here)

\hypertarget{graph-week-4}{%
\subsection{Graph Week 4}\label{graph-week-4}}

Question I hope to answer: (probably the same as previous week but may
change slightly)

Instructor's feedback: (copy the feedback from moodle here)

\hypertarget{graph-week-5}{%
\subsection{Graph Week 5}\label{graph-week-5}}

Question I hope to answer: (probably the same as previous week but may
change slightly)

Instructor's feedback: (copy the feedback from moodle here)

\end{document}
